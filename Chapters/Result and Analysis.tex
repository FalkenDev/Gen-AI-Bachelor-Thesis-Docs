%%%%%%%%%%%%%%%%%%%%%%%%
% Results and Analysis | Chapter 4
%%%%%%%%%%%%%%%%%%%%%%%%
\chapter{Results and Analysis}
\label{chp:results}
\section{On the content}
Your results and analysis, along with your discussion, will form the highlight of your thesis. This is where you report your findings and present them in a systematic manner. The expectations of the reader have been built up through the other chapters, make sure you fulfill these expectations.

As already described in Section~\ref{sec:theis-structure}, you should keep the results (the data you collected), the data analysis, and the interpretation/discussion of the results and the analysis separate, if possible. In many practical cases, it makes sense to describe the data and their analysis together.
%To analyse means to distinguish between different types of phenomena -- similar from different. Importantly, by distinguishing between different phenomena, your theory is put to work.
Precisely how your analysis should appear, however, is a methodological question. Finding out how best to organise and present your findings may take some time. A good place to look for examples and inspiration is journal publications or repositories for theses (e.g., DiVA). Regarding the latter, please keep in mind that repositories for theses contain all passed theses -- even the ones that received low grades. 

In the results, you typically provide descriptive statistics about your datasets and their properties, like populations, samples, and distributions.
Data visualizations, like histograms, can be very useful to present the results. In the analysis, you often use inferential statistics (among other things) to make inferences from the results (like generalizations and predictions).