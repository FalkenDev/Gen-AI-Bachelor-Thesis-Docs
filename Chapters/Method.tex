%%%%%%%%%%%%%%%%%%%%%%%%
% Method | Chapter 3
%%%%%%%%%%%%%%%%%%%%%%%%
\chapter{Method}
\label{chp:method}

\section{Research questions}
Main question
\begin{itemize}
    \item RQ1. Can the use of generative AI help improve the user experience in a data-intensive web application compared to traditional user interface methods?
\end{itemize}

Sub questions
\begin{itemize}
    \item RQ1.1. To what extent can generative AI improve the usability of a UI?

    With this research question we aim to explore the understandability of the UI. We will measure understandability by accounting for pondering time, and users asking for clarifications.

    \item RQ1.2. What are the key challenges for non-technical users using generative AI for data visualization in web applications?

    \item RQ1.3: Can generative AI help to improve perceived efficiency and task completion when working with a data-intensive web application?

    With this research question, we aim to explore efficiency in terms of tasks completed per time but also the perceived efficiency. We will measure efficiency by accounting for time spent per task and total tasks completed. The perceived efficiency will be measured through qualitative data collected in the exit interview.
    
\end{itemize}

\section{Literature review}
Search Strings used on Scopus
In order to locate the most relevant studies for our research on the impact of generative AI on user interfaces in data-intensive web applications, we have designed a search strategy to be employed within Scopus. The following search strings were used:

\begin{itemize}
    \item TITLE-ABS-KEY ( ( "generative AI" OR "generative artificial intelligence" ) AND ( "user interface" OR accessibility OR usability ) ) AND ( LIMIT-TO ( PUBSTAGE , "final" ) ) AND ( LIMIT-TO ( SUBJAREA , "COMP" ) OR LIMIT-TO ( SUBJAREA , "ENGI" ) OR LIMIT-TO ( SUBJAREA , "SOCI" ) OR LIMIT-TO ( SUBJAREA , "DECI" ) OR LIMIT-TO ( SUBJAREA , "BUSI" ) OR LIMIT-TO ( SUBJAREA , "ENVI" ) ) AND ( LIMIT-TO ( LANGUAGE , "English" ) )

    \item TITLE-ABS-KEY ( ( "data-intensive" OR "data intensive" ) AND "Web application" AND ( "user interface" OR accessibility OR usability ) ) AND ( LIMIT-TO ( PUBSTAGE , "final" ) ) AND ( LIMIT-TO ( SUBJAREA , "COMP" ) OR LIMIT-TO ( SUBJAREA , "ENGI" ) OR LIMIT-TO ( SUBJAREA , "SOCI" ) OR LIMIT-TO ( SUBJAREA , "DECI" ) OR LIMIT-TO ( SUBJAREA , "BUSI" ) OR LIMIT-TO ( SUBJAREA , "ENVI" ) ) AND ( LIMIT-TO ( LANGUAGE , "English" ) )
\end{itemize}

\section{Data Collection}
We are using a mix of different methods to collect data. These include a demographic survey, a “thing” and an exit interview.

To collect data for our “thing” we are using a mix of different methods. We are gathering quantitative data through our demographic survey, where we get information about our participants and the experience.

To collect data “that is relevant to the whole thing” we are going to perform an “experiment/observation thing”, where our participants perform data analysis tasks on a web application. Half of the participants will use a traditional application and the other half will also have access to a generative AI(that has access to the data the application has).

During the “thing”, the participants are prompted to verbalize their thoughts(thinking aloud), this will help us get as much data as possible from the few participants we are able to perform the experiment with. After the “thing” we will gather more qualitative data from an exit interview.

\subsection{Demographic Survey(Google Forms):}
To be performed before the session. With the intent of gathering information about our participants to understand their background and experience with technology, their occupation aswell as age. We will also inform the participants what the experiment will entail, including information about how the experiment will work, that they will be recorded and how the collected data will be used.

\begin{itemize}
    \item Inform and Get Consent: We are going to clearly explain the purpose of the study, what participation involves, risks, benefits, and confidentiality of responses. We will also ensure that our participants understand their rights, including withdrawal at any time without penalty. Obtain written or digital consent.
\end{itemize}
\begin{itemize}
    \item Age: To understand the age distribution of our participants.
    \item Experience: Gauge the overall experience level with technology, specifically web applications and AI tools.
    \item Experience in What: Understand the domains of their experience, e.g., software development, design, data analysis, etc.
    \item Occupation Information: Collect data on their current job roles to understand their professional background.
    \item Inform About the Observation: Detail what the observation will entail, including the tasks to be performed, the use of recording equipment (if any), and how the collected data will be used.
\end{itemize}

\subsection{Session preparation (On-site)}

To be done before each session/participant. We will setup the application, with or without the AI component depending on the treatment for that participant. Have the necessary programs running and making sure it looks the same for each participant. Tasks will be prepared and printed out so the participant can read at their own discretion if necessary.