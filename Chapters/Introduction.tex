%%%%%%%%%%%%%%%%%%%%%%%%
% Introduction | Chapter 1
%%%%%%%%%%%%%%%%%%%%%%%%
\chapter{Introduction}
\label{chp:introduction}  % labels are used for cross references

\section{Background}
Today, technological developments are advancing at an increasingly rapid pace, opening up new opportunities and challenges in software engineering. A particularly interesting and critical research area that is emerging is generative AI (artificial intelligence). The technology has caught the interest of many companies that see the potential of integrating generative AI into their web applications. Through this, user interfaces can become more intuitive and accessible, which can significantly improve the user experience.

According to an IBM Research Blog post \cite{ibmGenerativeAI} Generative AI refers to deeplearning models that can take raw data 

According to IBM Research Blog \cite{ibmGenerativeAI} Generative AI refers to a suite of models and techniques capable of creating new data that mimics original datasets based on the data it has been trained on. This allows a Generative AI to perform various tasks such as answering questions, composing texts, summarizing texts, and more, by generating text that is difficult to distinguish from text written by humans.

An example of a company exploring this technology is Connectitude AB. They are particularly interested in how generative AI can be applied in their data-intensive web applications to simplify the analysis and interpretation of large amounts of data for users. In our bachelor thesis, in collaboration with Connectitude AB, we will explore how generative AI can influence the user experience in web applications that require extensive data processing.

\begin{comment}
    Our goal is to identify both the positive and negative aspects of employing generative AI in settings that demand significant data handling, particularly those relating to certain aspects of user experience, such as usability(learnability, understandability) and efficiency.
\end{comment}

\section{Motivation and Value}
Since the integration of generative AI in applications is very new and not so explored, it is of interest to investigate the value of generative AI in data-intensive web applications. In addition, by highlighting the limitations and challenges of using generative AI in environments that require extensive data processing. It can lead to a better understanding of the current capabilities and areas for improvement, specifically for example usability concerning learnability and understandability. The research can also lay the groundwork for future research on how to go about implementing a Generative AI to your application and its complications there may be.

\begin{comment}
    Kanske nämna om att det kan vara bra att veta om man jobbar bättre och får bättre resultat när man använder generativ ai ?
\end{comment}

Data presented in a data-intensive web application can be very difficult to interpret for a user, but with the help of generative AI, the interpretation of data can become more user-friendly. This can help increase usability and value compared to previous methods in traditional graphical user interfaces. By integrating generative AI into a data-intensive web application, we believe it can offer users greater learning ability and reduce the time spent pondering what to do next when dealing with a data-intensive application, especially for those who are not as technical.

The article "Generative AI at work"\cite{generativAIAtWork} addresses whether using generative AI increases productivity for employees but also improvements in problem solving for those who are new workers or less technical. Providing generative AI can simplify information access for people with different levels of knowledge and make it more user-friendly by typing to a chatbot instead of looking around the application and trying to find the information you are looking for. This approach can also reduce the steep learning curve traditionally associated with these applications.

\section{Scope}
Our scope is to explore the use of generative artificial intelligence in the user experience and data management of web applications. By exploring generative AI in the user interface of a data-intensive web application, we want to understand and analyze its impact on the user experience.

The study focuses on how generative AI affects usability in terms of learnability and understandability. The study also focuses on how one's perceived confidence and efficiency is when performing tasks when using generative AI versus not. A further focus area is to assess the quality and relevance of the information generated by generative AI. We want to explore how well generative AI can analyse and adapt data to provide more insightful and relevant information.

\begin{comment}
Through this research, we hope to contribute insights into how generative AI can be used to improve user experience in data-intensive applications, and how these techniques can be adapted to meet the needs of a broad user base.

Our study aims to put generative AI in a broader social context, to understand its potential and limitations in terms of user experience in data-intensive web applications.
\end{comment}

\section{Outline}
Outline text here
\begin{comment}
    Kort förklaring om hur dokumentet är strukturerat och vad varje kapitel kommer  handla om, ifall jag fattat det rätt.
\end{comment}