%%%%%%%%%%%%%%%%%%%%%%%%
% Introduction | Chapter 1
%%%%%%%%%%%%%%%%%%%%%%%%
\chapter{Introduction}
\label{chp:introduction}  % labels are used for cross references

\section{Background}
Today, technological developments are advancing at an increasingly rapid pace, opening up new opportunities and challenges in software engineering. A particularly interesting and critical research area that is emerging is generative AI (artificial intelligence). This technology has caught the interest of many companies who see the potential of integrating generative AI into their web applications. Through this, user interfaces can become more intuitive and accessible, which can significantly improve the user experience.

One example of a company exploring this technology is Connectitude AB. They are particularly interested in how generative AI can be applied in their data-intensive web applications to simplify the analysis and interpretation of large amounts of data for the users. In our bachelor thesis, in collaboration with Connectitude AB, we will explore how generative AI can influence the user experience in web applications that require extensive data processing. Our goal is to identify both the positive and negative aspects of employing generative AI in settings that demand significant data handling, particularly those relating to certain aspects of user experience, such as usability(learnability, understandability) and efficiency.

\section{Motivation and Value}
With this research we want to inform the public and businesses about the value of generative AI in data-intensive web applications. Also by highlighting the limitations and challenges of using generative AI in environments that require extensive data processing. We it can lead to a better understanding current capabilities and areas for improvement, usability concerning learnability and understandability.

Data presented in a data-intensive web application can sometimes be very difficult to interpret for a user, but with the help of generative AI, data can be made more understandable and user-friendly. This can help increase usability and value compared to previous methods. By integrating generative AI into a data-intensive web application, we believe it can offer users greater learning ability and reduce the time spent pondering what to do next when dealing with a data-intensive application, especially for those who are not so technical.

The article "Generative AI at work"  addresses whether using generative AI increases productivity for employees but also improvements in problem solving for those who are new workers or less technical. Providing generative AI can simplify information access for individuals with different levels of knowledge and make it more user-friendly by typing to a chatbot instead of looking around the application and trying to find what you are looking for. This approach can also reduce the steep learning curve traditionally associated with these applications.

\section{Scope}
In our bachelor thesis in software engineering, we want to explore the use of generative artificial intelligence in user experience and data management in web applications. By exploring generative AI in a user interface of a data-intensive web application, we want to understand and analyze its impact on the user experience.

Our study focuses on the impact of generative AI on usability in learning ability and understandability. We also aim to explore how generative AI is perceived by different skilled users. A further focus area is to assess the quality and relevance of the information generated by generative AI. We want to explore how well generative AI can analyze and adapt data to provide more insightful and relevant information.

Through this research, we hope to contribute insights into how generative AI can be used to improve user experience in data-intensive applications, and how these techniques can be adapted to meet the needs of a broad user base.

Our study aims to put generative AI in a broader societal context, to understand its potential and limitations in terms of user experience in web applications.

\section{outline}
Outline text here
