%%%%%%%%%%%%%%%%%%%%%%%%
% Related work | Chapter 2
%%%%%%%%%%%%%%%%%%%%%%%%
\chapter{Related Work}
\label{chp:relatedwork}

\begin{comment}
   - Skirva att "Enhancing User Experience Journey: Simplifying Digital Interactions with Generative AI Integration" \cite{resGateGenAI} gör liknande sak men ser inget om att de har studerat om det faktiskt funkar utan mer tagit upp en ide ba. Sen kanske ta med Generative AI at Work \cite{generativAIAtWork} för de har ju använt ai i sitt arbete.
 
\end{comment}

This chapter presents work that relates in various ways to this bachelor thesis. The relationship is described and the difference between this work and the respective work is presented.

\section{Impact of Generative AI on Workplace Productivity}
The “Generative AI at work” \cite{generativAIAtWork} provides a comprehensive analysis of the role of Generative AI in improving customer service agents' productivity at work. Their findings from 5,179 customer service workers show that using Generative AI can significantly improve productivity. On average, their productivity increased by 14\% when measuring the number of tickets resolved per hour and by 34\% for employees who have no or very little experience. Also, the article states that by analyzing agents' chats, they found that inexperienced workers communicate more as highly skilled agents when using AI. That by AI helping the worker with giving recommendations, the worker learns best practices based on the fact that their AI is trained on highly skilled and more experienced agents.

This study is relevant to our research, as it highlights the potential of Generative AI to improve efficiency and effectiveness in data-intensive environments where user interface design plays a crucial role in facilitating user interaction and data management. The difference between this article and our research is that the article's implementation is primarily designed for support agents to be able to respond to customer problems while we want to investigate whether the user experience can be improved when using generative AI in a data intensive application for an analyst by being able to find the right data points, being able to analyze these points, and being able to understand what the data mean.

\section{Enhancing User Experience Through Generative AI}
The research paper 'Enhancing User Experience Journey: Simplifying Digital Interactions with Generative AI Integration' \cite{resGateGenAI} discusses the integration of a generative AI, such as ChatGPT, into digital platforms to potentially improve user experience by simplifying interactions and personalizing content. The study focuses on solving the problem that visitors find it difficult to navigate trough menus, sub menus, and filtering, among other things. The study underscores the potential of generative AI to eliminate complex navigation processes and improve user satisfaction through responsive and customized digital interfaces.

This research relates to some of our work on improving user interfaces in data-intensive web applications by demonstrating the practical benefits of generative AI to simplify user interactions and improve accessibility. Even so, we use an approximate approach to their solution for our experiment where we train a generative AI that acts as an intermediary between users and the website's interface.

However, the research paper has not produced any statistics to prove that it simplifies user interactions and improves accessibility, but presents more potential benefits and use cases, which is also the main difference between the paper and our research work.

\section{Conclusion}
The studies reviewed here demonstrate the benefits of generative AI, from improved user experience to increased productivity in the workplace. Our research aims to reveal specifically in data-intensive environments whether there are benefits in the user experience of using generative AI, thus filling a gap in the current literature on the subject.
