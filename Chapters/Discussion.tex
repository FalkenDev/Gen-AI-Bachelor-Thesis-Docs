%%%%%%%%%%%%%%%%%%%%%%%%
% Discussion | Chapter 5
%%%%%%%%%%%%%%%%%%%%%%%%
\chapter{Discussion}
\label{chp:discussion}
\begin{comment}
\section{On the content}
In many theses, the discussion is the most important section. Make sure that you allocate enough time and space for a good discussion. This is your opportunity to show that you have understood the significance of your findings and that you are capable of applying theory in an independent manner.

The discussion will consist of argumentation. In other words, you investigate a phenomenon from several different perspectives. To discuss means to consider different interpretations of your findings. Here are a few examples of formulations that signal argumentation:

\begin{itemize}
    \item On the one hand \dots and on the other \dots 
    \item However \dots
    \item \dots it could also be argued that \dots
    \item Another possible explanation may be \dots
\end{itemize}


\section{Things to keep in mind when writing your discussion}
\begin{enumerate}
    \item Try to structure your discussions from the ``specific'' to the ``general'': expand and transition from the narrow confines of your study (based on your results and analyses) to the general framework of your discipline.
    \item Make a consistent effort to stick with the same general tone of the introduction. This means using the same key terms, the same tense, and the same point of view as used in your introduction.
    \item Start by re-stating your research questions and/or hypotheses. Then declare the answers to them -- make sure to support these answers with a clear line of evidence that can be traced to your analysis and/or your data.
    \item Continue by explaining how your results relate to the expectations of your study and to the literature. Clearly explain why the results are trustworthy (or not if there are doubts about the reliability or validity of the data collection and/or analysis) and how they relate (supporting or contradicting) with previously published knowledge about the subject. If your thesis closes a research gap you or someone else has identified and described, state that explicitly. Make sure to use relevant citations.
    \item Make sure to give the proper attention to all the results relating to your research questions, this is regardless of whether or not the findings were statistically significant.
    \item Don't forget to tell your audience about the patterns, principles, and key relationships shown by each of your major findings and then put them into perspective. The sequencing of this information is important: 1) state the answer, 2) show the relevant results and 3) cite the work of credible sources. When necessary, point the audience to figures and/or graphs to ``enhance'' your argument.
    \item Make sure to justify your answers. Try to do so in two ways: by explaining the validity of your answer and by showing the potential shortcomings of others' answers. You will make your point of view more convincing if you provide both sides to the argument.
    \item Also make sure to identify conflicting data in your work. Make a good point of discussing and evaluating any conflicting explanations of your results. This is an effective way to win over your audience and make them sympathetic to any true knowledge your study might have to offer.
    \item Make sure to include a discussion of any unexpected findings. When doing this, begin with a paragraph about the finding and then describe it. Also, identify potential limitations and weaknesses inherent in your study. Then comment on the importance of these limitations to the interpretation of your findings and how they may impact their validity. Do not use an apologetic tone in this section. Every study has limitations. Showing your awareness regarding the limitations increases the trustworthiness of your reasoning.
    \item Conduct a brief summary of the principal implications of your findings for research and practice in the area (do this regardless of any statistical significance). You might also want to provide recommendations for potential research in the future related to your implications. A brief summary of these recommendations will typically be included in Chapter~\ref{chp:conclusions} (\emph{\nameref{chp:conclusions})}.
    \item You should also think about the potential wider consequences of your results. Do they have an impact on society? What are the possible ethical consequences of your findings? Does your work have an impact on sustainability (in whatever dimension)?
%    \item Show how the results of your study and their conclusions are significant and how they impact our understanding of the problem(s) that your thesis examines. This will make the contribution of your thesis more explicit.
    \item On a final note, discuss everything that is relevant \emph{but be brief, specific, and to the point}.
\end{enumerate}
\end{comment}


